% Created 2018-04-30 Mon 01:35
% Intended LaTeX compiler: pdflatex
\documentclass[11pt]{article}
\usepackage[utf8]{inputenc}
\usepackage[T1]{fontenc}
\usepackage{graphicx}
\usepackage{grffile}
\usepackage{longtable}
\usepackage{wrapfig}
\usepackage{rotating}
\usepackage[normalem]{ulem}
\usepackage{amsmath}
\usepackage{textcomp}
\usepackage{amssymb}
\usepackage{capt-of}
\usepackage{hyperref}
\usepackage[margin=1.5in]{geometry}
\author{Tony Wang}
\date{Fri April 27th}
\title{Investigating Parallel Programming Methods in Notebook-based Programming Environments}
\hypersetup{
 pdfauthor={Tony Wang},
 pdftitle={Investigating Parallel Programming Methods in Notebook-based Programming Environments},
 pdfkeywords={},
 pdfsubject={},
 pdfcreator={Emacs 25.2.1 (Org mode 9.0.7)}, 
 pdflang={English}}
\begin{document}

\maketitle

\section{Summary Paragraph}
\label{sec:org53f6988}
Notebook-based programming (programs that combine data,
visualizations, and code) has become increasingly popular over the
past few years, specializing in making the data exploration and
initial development processes easier. However, they are poorly suited
for high performance computing and a typical "data science" workflow
currently requires a second step of rewriting the code in a parallel
manner or in an entirely different environment/language. This project
will explore a myriad of parallel programming tools available in
Python and Jupyter with the specific aim of helping to inform
developers in what cases would a notebook-based parallel programming
solution suffice or whether a refactor is required. Approaches to
explore include Ipython's parallel \%magic functions, Cython
code optimizations, IPython Parallel, Dask package. These approaches
will be evaluated on various machine learning tasks with well-defined
benchmarks, such as the Iris flower dataset or the Boston housing
market dataset. A final test on a real-world machine learning workflow
will be conducted using data from Electronic Health Records (EHR) to
predict sepsis.

\section{Team}
\label{sec:org58e4433}
\begin{itemize}
\item Tony Wang
\item Lawrence Wolf-Sonquim
\end{itemize}
\end{document}
